\documentclass[a4paper,11pt,english]{article}
\usepackage[english]{babel} 
\usepackage[T1]{fontenc}    
\usepackage[utf8]{inputenc} 
\usepackage{graphicx}       
\usepackage{hyperref}      


\begin{document}

\title{Master project: final report template}
\author{Mikko Lauri \and Ge Gao}
	
\maketitle 

\section{Introduction}
Use this template as a starting point for preparing your final report.

The recommended length of the final report should be at most 20 pages.
The report should contain the following sections:
\begin{enumerate}
	\item Introduction, which describes the motivation and goal of your project. Describe how your project is connected to existing literature. Include references to scientific papers that are relevant to your topic. Do not copy your text from the intermediate report. Revise and rewrite according to the feedback you got and what the actual outcome of the project was.
	\item Theoretical background, which provides the descriptions on what existing theories and ideas that are related to your problem statement. The purpose of this section is to ensure that you are knowledgeable about the related key concepts, theories and models. Do not copy your text from the intermediate report. Revise and rewrite according to the feedback you got and what the actual outcome of the project was.
	\item Implementation, which connects with the section on theoretical background and describes how the required parts were implemented, and how the overall system is constructed.
	\item Results, which contain the descriptions of the experiments conducted and presents their results.
	\item Conclusions, which summarizes the work that was done and the results obtained, and gives suggestions for future work. \textbf{Most importantly}, include a section that contrasts the actual outcome of the project to the plan you submitted: what was planned, what was actually achieved, what were the main reasons for the actual outcome, what you consider you have learned during the project, etc.
\end{enumerate}

For more information on \LaTeX, please consult, e.g., the online book at \url{https://en.wikibooks.org/wiki/LaTeX}.
Refer also to material on scientific writing.


\end{document}
