%!TEX root = ../poster.tex

\begin{columns}[t,totalwidth=\twocolwid]

\begin{column}{\onecolwid}

	\vspace{1cm}

	\large{
		\begin{itemize}\itemsep48pt
			\item Gazebo simulation using the Yale-CMU-Berkeley object set\cite{calli2015benchmarking}.
			\item System running using the Robot Operating System (ROS)
		\end{itemize}}

\end{column}

\begin{column}{\sepwid}
\end{column} % Empty spacer column

\begin{column}{\onecolwid}
	% \begin{itemize}
	\large{
		Five next best view calculation methods compared:
			\begin{itemize}
				\item \textbf{Random} selection
				\item \textbf{FBE}: frontier-based exploration
				\item \textbf{SBE (obs)}: sampling-based exploration with only obstacle information
				\item \textbf{SBE (obs+cands)}: sampling-based exploration with obstacle and candidate information
				\item \textbf{FBE plus}: starting with FBE, then switching to SBE (obs)
			\end{itemize}}
	% \end{itemize}
\end{column}
\end{columns}

%\begin{itemize}
%	\item \texttt{Random} choice
%	\item Frontier-based exploration (\texttt{FBE}), which seeks to simply view all unseen space as quickly as possible
%	\item Sampling-based exploration, which samples the known free space and calculates the information gain at each sampled point based on only known obstacles (\texttt{SBE (obs)}) or both obstacles and candidates (\texttt{SBE (obs+cands)})
%	\item Frontier+ (\texttt{FBE plus}), which uses \texttt{FBE} in the initial exploration phase, then switches to \texttt{SBE (obs)}
%\end{itemize}

%Sampling-based methods calculated information gain by raycasting from each candidate viewpose, and scoring the pose based on the types of voxels visible from it: unknown voxels scored 2 points, obstacles 10, and unknown voxels next to objects 20.
%Additionally, SBE (obs+cands) awarded extra bonuses for voxels containing candidates and unknown voxels next to candidates.