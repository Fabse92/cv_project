%!TEX root = ../poster.tex

Experiments were conducted in the Gazebo simulator using the YCB object set\cite{calli2015benchmarking}.
Five NBV calculation methods were implemented: 
\begin{itemize}
	\item \texttt{Random} choice
	\item Frontier-based exploration (\texttt{FBE}), which seeks to simply view all unseen space as quickly as possible
	\item Sampling-based exploration, which samples the known free space and calculates the information gain at each sampled point based on only known obstacles (\texttt{SBE (obs)}) or both obstacles and candidates (\texttt{SBE (obs+cands)})
	\item Frontier+ (\texttt{FBE plus}), which uses \texttt{FBE} in the initial exploration phase, then switches to \texttt{SBE (obs)}
\end{itemize}
Sampling-based methods calculated information gain by raycasting from each candidate viewpose, and scoring the pose based on the types of voxels visible from it: unknown voxels scored 2 points, obstacles 10, and unknown voxels next to objects 20.
Additionally, SBE (obs+cands) awarded extra bonuses for voxels containing candidates and unknown voxels next to candidates.